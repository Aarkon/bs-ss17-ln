\documentclass[]{scrartcl}
\usepackage[utf8]{inputenc}
\usepackage{ngerman}
\usepackage{listings}
%\lstset{basicstyle=\ttfamily,
%	showstringspaces=false,
%	commentstyle=\color{red},
%	keywordstyle=\color{blue}
%}

%opening
\title{Betriebssysteme Praktikum 1}
\author{Florina Nehmer\\
Jakob Ledig}


\renewcommand{\thesubsection}{\thesection.\alph{subsection}}
\begin{document}

\maketitle

\section{Bash}

\subsection{„Erste Erfahrungen mit der Bash“}
Hier folgt ein kleiner Abriss über die wichtigen Kommandos

\subsection{Fragen beantworten}
\subsubsection*{Variablen}
\paragraph{\$HOME}: Link aufs Homeverzeichnis des aktuell eingelogten Users.
\paragraph{\$PATH}: In den hier vorgehaltenen Pfaden abgelegte (Binär-)dateien werden ausgeführt, wenn ihre Name in der Bash aufgerufen wird.
\paragraph{\$UID}: Gibt die ID des aufrufenden Users aus. Wird beginnend mit 1000 numerisch fortlaufend vergeben.
\paragraph{\$USER}: Name des aufrufenden Users.

\subsubsection*{\$HOME}
Der Befehl wechselt ins Homeverzeichnis des Users. Kürzer ist:
\begin{lstlisting}[language=bash]
	cd ~
\end{lstlisting} .

\subsubsection*{Pfeilasten und STRG + D}
$\uparrow$ geht in der Befehlshistorie rückwärts, $\downarrow$ vorwärts. STRG + D schließt das Terminal und beendet die Sitzung.

\subsubsection*{.bashrc}
bashrc ist ein Shellscript, dessen Inhalt beim Öffnen eines Terminalfensters ausgeführt wird.

\section{Bash-Script}
\subsection{frename.sh}
Siehe Quellcode

\subsection{try\_host.sh}
Siehe Quellcode

\subsection{ausführbar machen}
chmod\footnote{Als root auszuführen, d.h. je nach Distribution entweder als Benutzer root oder mit vorangestelltem „sudo“.} +x \$Dateiname, danach ist das Skript ausführbar.

\subsubsection{\$PATH erweitern}
Kein Plan was gemeint ist, nachfragen!

\end{document}
