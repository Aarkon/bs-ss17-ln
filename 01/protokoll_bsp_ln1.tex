\documentclass[]{article}
\usepackage{inputenc}[utf8]
\usepackage{ngerman}

%opening
\title{Betriebssystem Praktikum 1}
\author{Florina Nehmer\\
Jakob Ledig}

\begin{document}

\maketitle

\section{Bash}
\subsection{„Erste Erfahrungen mit der Bash“}

\subsection{Fragen beantworten}
\subsubsection{Variablen}
\paragraph{\$HOME}: Link aufs Homeverzeichnis des aktuell eingelogten Users.
\paragraph{\$PATH}: In den hier vorgehaltenen Pfaden abgelegte (Binär-)dateien werden ausgeführt, wenn ihre Name in der Bash aufgerufen wird.
\paragraph{\$UID}: Gibt die ID des aufrufenden Users aus. Wird beginnend mit 1000 numerisch fortlaufend vergeben.
\paragraph{\$USER}: Name des aufrufenden Users.

\subsubsection{\$HOME}
Der Befehl wechselt ins Homeverzeichnis des Users. Kürzer ist cd ~.

\subsubsection{}
$\uparrow$ geht in der Befehlshistorie rückwärts, $\downarrow$ vorwärts. STRG + D schließt das Terminal und beendet die Sitzung.

\subsubsection{.bashrc}
bashrc ist ein Shellscript, dessen Inhalt beim Öffnen eines Terminalfensters ausgeführt wird.

\end{document}
